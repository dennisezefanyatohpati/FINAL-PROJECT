\justifying
\noindent

\section{Introduction}
\subsection{Formula Student}
Formula student United Kingdom (FSUK) is an annual motor-sport engineering competition which held by Institute of Mechanical Engineers (IMechE). Formula student was first run by its inception, The Society of Automotive Engineers (SAE) in 1981 in the United States of America which known as Formula SAE. Every July since 2007, hundreds of universities from around the world gather at Silverstone Circuit for the final competition. The objective of every team is to design and build a single-seat race car which then will be judged from number of engineering and business aspects.

The competition mainly consist of 2 events: static and dynamics. The static event judges the Engineering  Design,  Cost  and  Sustainability  Analysis, Business Presentation and Technical Inspection, and the dynamic event judges Skid Pad, Sprint, Acceleration, Endurance, and Fuel Economy.

Queen's University Belfast is represented by Queen's Formula Racing (QFR) in the event of FSUK since 2001. The team is consisted of undergraduates and postgraduate students which are divided into Engineering sub-teams focusing on chassis, suspension \& unsprung mass, powertrain, electrical, and performance. Consideration of aerodynamic aspect on QFR car was first initiated in 2017 which focus on the aerodynamic analysis framework. In 2018 the first design of aerodynamics undertray for QFR race car was done, and this paper is intended to build up the analyses from previous paper and improve the undertray performance by investigating broader variables which could enhance the overall car performance.  

\subsection{The significance of Aerodynamics}



\subsection{Project Aims}
The aim of this project is to design and optimise the aerodynamics undertray for Queen's Formula Racing car which improve the car's overall down-force and drag reduction. Final design of the undertray will be based on the computational simulations and design recommendation from previous QFR analysis. The approved final design then will be manufactured and installed to QFR 2021 for the competition if the workshop time and capacity allows.
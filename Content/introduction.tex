\newpage
\setcounter{page}{1}
\justifying
\noindent

\section{Introduction}
\subsection{Formula Student}
Formula Student United Kingdom (FSUK) is an annual motor-sport engineering competition held by the Institute of Mechanical Engineers (IMechE). Every July since 2007, hundreds of universities from around the world gather at the Silverstone Circuit for the final competition. Every team's objective is to design and build a single-seat race car, which will be judged from several engineering and business aspects. The competition mainly consists of two events: static and dynamic. The static event judges the Engineering  Design,  Cost, and  Sustainability  Analysis; Business Presentation; and Technical Inspection, and the dynamic event evaluates Skid Pad; Sprint; Acceleration; Endurance; and Fuel.

\begin{figure}[!ht]
\begin{center}
%    
  \begin{subfigure}[b]{0.45\textwidth}
    \includegraphics[height=4.5cm]{Figures/QFR17PHOTO.JPG}
  \end{subfigure}
  %
  \begin{subfigure}[b]{0.45\textwidth}
    \includegraphics[height=4.5cm]{Figures/QFR18PHOTO.jpg}
  \end{subfigure}
%  
  \caption{Queen's Formula Racing car 2017 (left) and 2018 (right) on the dynamics event}
    \label{fig:1}
\end{center}
\end{figure}

\noindent Queen's Formula Student first initiated consideration of aerodynamic aspects on the QFR car in 2017, which focused on the aerodynamic analysis framework \cite{Corr2017MechanicalAuthor}. In 2018, the first design of an aerodynamic undertray for the QFR race car was generated \cite{McKeown2018DesignCar}. This paper is intended to build up the analyses from previous projects and improve the undertray performance by investigating broader and deeper variables that could plausibly enhance the overall car performance. 

\subsection{The Significance of Aerodynamics in Formula Industry}
Aerodynamics has become a crucial aspect of high-speed car performance. The race-car industry has led technology innovation by indicating the need for constant improvement \cite{Zhang2006GroundCars}. Engineers have been striving to sculpt the shape of the cars to manipulate and take advantage of the flow round the body. The role of aerodynamics in improving race-car performance rose to prominence in 1968, when an inverted airfoil was first introduced to a Formula One car, and the research in this area has been growing exponentially ever since.

%put inverted aerofoil race car photo here

\noindent Downforce(negative-lift) is an important aerodynamic effect, and is key in improving overall car performance. Downforce, or negative lift, is the force produced by the flow of air around the body. This is usually achieved on a high-speed ground vehicle by introducing aerodynamic devices such as wings and undertrays, which modify the airflow to suit the engineering needs \cite{Wright1982TheCars}. In the race-car industry, the primary aim is to maximise the downforce while maintaining the lowest possible drag \cite{Zhang2006GroundCars}, but achieving consistent performance at diverse speeds and maximising acceleration is equally essential. The magnitude of the downforce significantly affects braking, acceleration, and cornering --- and hence the cornering speed. Despite the restrictive aerodynamics rules in the competition, optimising the downforce could improve the acceleration, increasing the chance to overtake the opponent on a corner. Higher downforce could also allow the top speed to be reached in a shorter time, reducing corner entry and exit time. 

\noindent However, the ground-effect aerodynamics that applies to an open-wheeled car is still an experimental science \cite{Zhang2006GroundCars}. This is due to the complex physics flow that involves a turbulent wake, the interaction of the ground boundary layer, dynamic suspension motion, and many more factors, in which accurate analytical capability (experimental \& computational) is not yet sufficient.

\noindent Nevertheless, computational fluid dynamics (CFD) has improved tremendously within the industry over the years and produced accurate results of forces, flow pattern, etc., in some particular car geometries. However, one research paper \cite{Zhang2006GroundCars} stated that race car diffuser is one of the most complex parts, which is hardly understood. Therefore, specific variables in generating an undertray geometry is required, with careful consideration of the assumptions on which the analysis is based.

\noindent This project will utilise CFD to understand the trends in behaviour of an aerodynamic undertray in both two and three dimensions, with several nozzle \& diffuser variables such as angle, length, and width. These results will then be used to determine the geometry of the final aerodynamic undertray for QFR 2021.

\subsection{Project Aims \& Objectives}
This project aims to design and optimise the aerodynamic undertray for the Queen's Formula Racing car, to improve the car's overall downforce and to reduce its. The undertray's final design will be based on new computational simulations as well as design recommendations from previous QFR analysis. The approved final design then will be manufactured and installed to QFR 2021 for the competition, if the workshop time and capacity allows.

\noindent
The key objectives to fulfil this aim are:
\begin{itemize}

    \item Analyse the enclosed and open flow 2D \& 3D geometries with various inlet, outlet, and ground clearance variables using ANSYS Fluent, in order to identify their effects on the undertray lift and drag trends. 
    
    \item Design flexible 3D undertrays based on the 2D analysis results, which will again be optimised with additional aerodynamic features on the undertray, improving the downforce and reducing the drag.
    
	\item Design the final undertray CAD and choose the appropriate material tailored to the car’s dimension and goals, then analyse the weight to drag ratio to pick the best performing material for the car.
	
    \item Manufacture and fit the undertray for the 2021 Queen’s Formula Racing car, which then will be judged in the Formula Student competition.
\end{itemize}
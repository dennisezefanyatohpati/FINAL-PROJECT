\section{Conclusions}
\justifying
\subsection{2D Enclosed Flow}
The 2D enclosed flow analysis was done to obtain an initial and fundamental understanding of the flow behaviour and its effect on the undertray performance. From the results obtained, a clear trend could be seen. An increase in the diffuser angle resulted a linear increase in lift (a reduction in downforce) and in drag. This is due to the early boundary layer separation on the diffuser region which creates an immediate jump in wall shear stress and a pressure increase, hence the trend. On the other hand, an increase in the inlet angle resulted in a reduction in lift (an increase in downforce) and an increase in drag. This is due to the fact that effective flow acceleration was achieved in the inlet region, therefore a higher velocity was achieved at the throat, generating a significant pressure drop. Moreover, the increase in drag is caused by the imposed force of the flow and increase of the wall friction in the flow direction. It is worth noting that there is a major limitation in this approach, which is the non-existence of any of the surrounding flow field which affects the flow at the rear, intake, and outlet. These shortcomings inspired the next approach to the analysis.

\subsection{2D Open Flow}
Two dimensional open flow analysis was conducted to include the effect of the flow around the car, which couldn't be done in the enclosed flow approach. Three different geometries were used to analyse the flow at the inlet, outlet, and rear of the car. From observation, a significant increase in downforce occurred at a 4 degree diffuser angle, followed by a gradual linear degradation up to a diffuser angle of 30 degrees. For similar reasons to those seen in the enclosed flow, this was caused by an early separation of the boundary layer in the diffuser at higher outlet angles. However, no significant changes or pattern in drag were seen at any given angle. It was surmised that drag was generated by the bluff body's skin friction drag and form drag, which made changes in undertray's drag insignificant by comparison. On the other hand, increasing the inlet angle proportionally increased the downforce and drag (however insignificantly) but not so strongly as was seen in the enclosed flow. It was found that the stagnation point on the front of the bluff body moves, slowing down the initial intake flow and generating separation at the inlet, which reduces the effectiveness of the intake acceleration. A third geometry was generated with a smooth and longer diffuser to improve flow attachment on the diffuser surface, and a smooth intake to overcome the flow separation there. The results showed that the smoother and longer diffuser increased the downforce significantly, without high drag penalties. Although this approach showed the effects of the flow surrounding the bluff body affecting the car in a 2D manner, there still remain 3D features which may have a significant effect on the undertray's performance.

\subsection{3D Open Flow}
The 3D open flow analysis extends the 2D open flow analysis. The purpose of this simulation is to build an idea of the undertray's flow in a three dimensional manner. Geometry 3 from the 2D open flow analysis was extruded, with a skirt added to help maintain flow isolation on the diffuser region. It was found that the overall downforce and drag have similar trends, but with lower values. This was suspected to be caused by the flow suction effect from the side body, which reduced flow acceleration efficiency due to the pressure difference. Distinctively, the best performing inlet angle was found to be at 5 degrees, after which followed an increase in lift and drag. With an increase in inlet angle, the corner vortices generated on the inlet region are larger due to the amount of flow directed to the side of the body compared to underneath. Similarly to the 2D open flow simulation, the diffuser angle has the highest downforce at 5 degrees which then progressively degrades up to 20 degrees. This is due to flow separation, which is not eliminated by the corner vortices, reducing the downforce. To overcome this, fences or vortex generators were installed to improve the performance. Analysis shows that the fences generated vortices due to a pressure difference which allows suction from the side-body. The generation of extra vortices delayed flow separation in the diffuser's boundary layer, increasing the effectiveness of the undertray. Nonetheless, with this increase in downforce, small penalties in the overall drag were also found.

\subsection{Undertray Design with Bluff Body}
Previous analyses were used as an initial benchmark for the undertray design. Eight undertray prototypes were generated with various geometric features to improve the downforce and reduce drag. A bluff-body traced from the 2018 QFR car was utilised to clearly capture the flow feature of an undertray attached to the car. The bluff body itself was also simulated with the same setup, as a baseline for comparison. Among eight designs, Undertray Geometry 2 (UTP2, see figure \ref{fig:UTP2_FINAL_DESIGN} in Appendix B) was chosen, with the lowest lift-to-drag ratio at -3.26. It was found that vortices generated at the rear have to be stable to maintain high downforce and attachment of the streamline in the diffuser region. For designs with two variable diffuser angles, an increase in downforce is found due to the formation of a larger vortex on the outermost diffuser, which creates a stronger suction effect. However, the potential of flow separation of the boundary layer with a big vortex being generated is high. It was surmised that early flow separation generates unstable vortices at the rear which increase the overall drag. Moreover, an additional curved side-diffuser such as in UTP 2, 5, 6, and 7 increases flow acceleration due to the lateral curvature, providing a pressure reduction; this features allows an almost doubling of the lift-to-drag ratio compared to the rest of the designs. From the simulations, the final design was able to increase the downforce by 184.12 N (678.5\% improvement) and reduce drag by 10.49N (13\% drag reduction) at 40 km/h compared to the baseline model.


\section{Future Work}
\noindent Extensive research into both the 2D and 3D behaviour of an undertray has been conducted. It is recommended that, in future, only 3D simulations should be used from the outset, as exploring the 2D analyses from previous and the current work shows that some critical flow features were not properly captured. The generation of streamwise vortices was found to play an important point in downforce generation, and therefore it is crucial to investigate further the variables which influence this, such as: interval distance, angle of attack, height, and the effect due to gap clearance. The incorporation of a bluff body is the recommended path to follow when investigating the vortex flow features. A device such as an undertray is a very specialised and custom design. Every car has various geometries which affect the flow through the undertray and which in turn get affected by the undertray. Therefore, it is necessary to analyse the flow features on a bluff body in lieu of the prototype design and incorporate this into the undertray design process. This will allows better understanding of the flow features which then can be utilised for the future design of the car.


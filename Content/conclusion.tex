\newpage
\section{Conclusion}
\justifying
\subsection*{2D Enclosed Flow}
The 2D Open flow analysis was done to obtain an initial understanding of flow behaviour and its effect on the undertray performance. From the results obtained, there was a clear trend can be seen. An increase in diffuser angle resulted a linear increase in lift (reduction in downforce) and drag. This is due to the early boundary layer separation on the diffuser region which create an immediate jump in wall shear drop and pressure increase, hence the trend. On the other hand, an increase in inlet angle resulted an reduction in lift (increase in downforce) and increase in drag. This is due to the fact that effective flow acceleration was achieved at inlet region, therefore higher velocity achieved at high angle on the throat which generate significant pressure drop. Moreover the increase in drag is caused by the imposed force of the flow and increase of wall friction in the flow direction. It is worth to be noted that the major limitations which is the non-existence of surrounding flow which affect the flow at the rear, intake, and outlet which then resolved in the next analysis.


\subsection*{2D Open Flow}
2D Open-flow analysis was conducted to resolve surrounding flow presence from previous analyses. Three different geometries were used to analyse the flow at the inlet, outlet and rear. From observation, significant drop occurred at 4 degrees diffuser angle, followed by downforce degradation up to 30 degrees diffuser angle. Similar reasons, this was caused by an early separation of an early separation of boundary layer in the diffuser at higher angle. However no significant changes or pattern in drag at any given angle, this was surmised that drag was generated by the body's skin friction drag and force opposing the frontal area which make changes in undertray's drag insignificant. On the other hand, increase in inlet angle is proportional to its downforce and drag (however insignificant) but not as strong as enclosed flow. It was found that stagnation point in the forefront of bluff body slows down the initial intake flow and generate separation point at the inlet which reduce the effectiveness of the intake acceleration. Third geometry was generated with smooth and longer diffuser to improve flow attachment on the diffuser area, and smooth intake to overcome the flow separation. The results showed that smoother and longer diffuser increase the downforce significantly without high drag penalties. Despite the flow surrounding the bluff body affect the car in 2D manner, there are still 3D features which may have a significant affect on the undertray's performance.

\subsection*{3D Open-Flow}
3D Open flow is an extension from 2D open flow analysis. The purpose of this simulation is to obtain apprehension of undertray's flow in 3D manner. Geometry 3 from 2D open-flow was extruded with skirt added to help flow isolation on the diffuser region. It was found that overall downforce and drag have similar trend with lower value. This was suspected to the flow suction effect from the side body which reduced flow acceleration efficiency due to pressure difference. Distinctively, the maximum performance of inlet angle occurred 5 degrees which then followed by increase in lift and drag. With an increase of inlet angle, corner vortex generated on the inlet region are larger due to the amount of flow directed to the side of the body compared to underneath. Similar to the 2D open-flow simulation, the diffuser angle has the highest downforce at 5 degrees which then degrades up to 20 degrees. This is due to the flow separation which not anticipated by the corner vortices, hence reduction in downforce. To overcome this, fences or vortex generator were installed to improve the undertray's downforce. Analysis shows that fences generated vortex due to pressure difference which allows suction from the side-body generating more vortices. The generation of extra vortices delay flow separation in the diffuser's boundary layer, increasing effectiveness of the undertray. Nonetheless, With increase in downforce, small penalties in overall drag increase also applies.



\subsection*{Undertray Design with Bluff Body}


\subsection{Future Work}
\noindent An extensive research in both 2D and 3D of an undertray have been conducted. It is recommended in future work that only 3D analysis should be the initial benchmark analyses, as 2D from previous and current paper conclude that some flow features were not captured. A Vortex generator was found to play an important point in downforce generation, therefore it is crucial to investigate further its variables, such as; interval distance, angle of attack, height, and its effect due to gap clearance. A bluff body and dimensionless variables is a recommended path to investigate the vortex flow features.

A device such as an undertray is a very customised design, every race car has a different flow features which affects and get affected by the undertray. Therefore, It is better to analyse flow features on a bluff body in lieu to the prototype design and applies it into the undertray. This will allows better understanding of the flow features which then can be utilised for the future generation car.


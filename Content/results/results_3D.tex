\section{Three Dimensional Analysis}
\noindent This section extends the analysis of the two dimensional approach, and is intended to explore and analyse the  three dimensional flow features. It was surmised that three dimensional flow allows for more complex flow behaviour, that may significantly affect an undertray's overall performance. This part of the report will be divided into 3D open flow, and undertray prototype with bluff body analyses.

\subsection{3D Open Flow}
This analysis is an extension to three dimensions of the 2D open flow cases. The purpose of this analysis to investigate further the flow features that develop in an undertray in a three dimensional manner, which might plausibly affect its performance.

\subsubsection{Geometry and Mesh Generation}
An identical 2D sketch from geometry 3 of the 2D open flow section was used, which then extruded to a car-realistic 1 meter thickness. A skirt was added to both sides of the rear diffuser. These skirts were used to improve flow isolation and to generate corner vortices, which help to improve flow attachment within the diffuser region. Fences were also added to later analyses to investigate the effects of vortex generators with respect to the generation of downforce. A sketch of the geometry can be seen in Figure~\ref{fig:3D_OF_GEOM} below. 

\begin{figure}[!h]
    \centering
    \noindent\makebox[\textwidth]{
    \includegraphics[width=0.8\textwidth]{Figures/3D_OF/3D_OF_D.PNG}}
    \caption{Geometry generated for 3D open flow analysis.}
    \label{fig:3D_OF_GEOM}
\end{figure}

\noindent The CAD geometry was then imported into DesignModeler, where the fluid domain and body of influence were built. The fluid domain was made 12 meters behind, 5 meters forward, 2 meters top, 0.03 meters bottom, and 1.5 meters wide. Due to the symmetry of the body, a half model was used for the fluid domain to minimise the mesh element count. A body of influence (BOI), with dimensions of  $6 \times 1 \times 7$ meters was created using a box feature around the body to increase the mesh density and capture the surrounding flow features.  % NON DIMENSIONALISED IT HOW MANY CAR LENGTH

\noindent A hybrid mesh was used, which is comprised of tetrahedrons and triangular prisms. Twenty inflation layers of triangular prism were applied near the wall to give a $y^+=1$. A max element size of 0.8 meters and 0.02 meters in the BOI generated a mesh with 1.3 to 1.5 elements for this analysis. This section's mesh quality is considered acceptable, with an average skewness of 1.9 and an aspect ratio of 80-90; however, a significant jump in size in the cells between the inflation layer and far-field fluid domain is considered not ideal. Detailed illustration of the fluid domain and the mesh can be seen in Figure~\ref{fig:3D_OF_MESH} Appendix C.

\subsubsection{Results and Discussions}
Due to the nature of the mesh, the $k-\omega$ SST model was used throughout this analysis to take advantage of the features of this particular transport model \cite{Ansys2006ModelingFlows}. The analyses will be grouped into four sets of geometries: varying diffuser angle with no inlet angle, varying diffuser angles with an inlet angle of 10 degrees, varying inlet angle with 10 degrees diffuser angle, and variable diffuser angle with three fences applied. Figure~\ref{fig:3D_OF_PLOT_COMPARE_ALL} below shows the comparison of lift and drag between all the variables.

\begin{figure}[htb!]
    \centering
    \noindent\makebox[\textwidth]{
    \includegraphics[width=0.9\textwidth]{Figures/Graph/3D_OF.png}}
    \caption{Lift and drag variation of diffuser (left) and inlet (right) angle for all geometry configurations.}
    \label{fig:3D_OF_PLOT_COMPARE_ALL}
\end{figure}

\noindent The values of lift and drag as computed represent the values for the half body. Once more, a similar trend emerged as with the 2D open flow geometry 3. The trend shows an increase in downforce at a 5 degrees diffuser angle, followed by the downforce falling linearly up to 20 degrees. In comparison with 2D open flow analysis, the average downforce of this analysis is significantly lower. This is due to the nature of the analysis which allows the accelerated flow in the undertray to be affected by the flow surrounding the body. This is in contrast to the 2D open-flow, in which the flow past the undertray is constrained to move solely in plane. In three dimensions, the lower pressure region in the undertray sucks in the air from the surroundings, reducing the effectiveness of the flow acceleration underneath, resulting in a lower downforce. This phenomenon is illustrated in Figure~\ref{fig:Vector_suction_diagram} below, with the influence of the corner vortices occurring from the fore of the bluff body. 

\begin{figure}[!htb]
    \centering
    \noindent\makebox[\textwidth]{
    \includegraphics[width=0.8\textwidth]{Figures/3D_OF/3D_OF_VECTOR_SUCTION.png}}
    \caption{Vector diagram of velocity flow on the undertray's cross-section indicating flow suction from the side of the body.}
    \label{fig:Vector_suction_diagram}
\end{figure}

\noindent The inlet angle elevation was simulated with 10 degrees diffuser angle. It can be examined that the effect of inlet angle elevation shows significant difference to the 2 dimensional analyses. Lift and drag of the bluff body reach its minimum at 5 degrees which then followed by lift and drag rise up to 20 degrees. As discussed in earlier section, stagnation point in the forepart of bluff body in 2 dimensional simulation creates separation point of flow direction to the top and bottom. However, in 3 dimensional case, the stagnation point separates the flow without consistent direction which create more complex flow features in the inlet region. An identical front cross-section of the car was made on the fore-flow of the body which became the initial location of streamline. Figure \textbf{\ref{fig:3D_OF_INLET_COMPARE} left} shows the streamline separation from the free flow to the surrounding body. Moreover, some of the streamline flow are leaked to the side, creating a trailing corner vortex \textbf{(middle)} hence reducing the intake acceleration in the undertray.  Compared to the 2 dimensional analysis, the flow that goes to the undertray stays without leaking or generate trailing vortex, which make the flow intake better, hence higher downforce. This explains the downforce degradation after 5 degrees, as the inlet angle increases, more airflow are leaking on the inlet region and creating bigger trailing vortex. Increase in drag comes from the skin friction as the flow is imposed by larger area with elevation of angle, and the trailing vortex which plausibly affect the flow on the throat (as discussed previously). It is worth noting that the changes of drag with inlet angle elevation is not significant and can be ignored, hence it was not used in  bluff body geometry with fences.

\begin{figure}[!htb]
    \centering
    \noindent\makebox[\textwidth]{
    \includegraphics[width=1\textwidth]{Figures/3D_OF/3D_OF_INLET_COMPARE.png}}
    \caption{Fore-flow streamline imposing the bluff-body and occurrence of leaking indicated by the trailing vortex ($Q$-criterion isosurface) at the inlet region (left and middle), and velocity vector of the flow indicating flow leaks at the inlet region (right).} 
    \label{fig:3D_OF_INLET_COMPARE}
\end{figure}

\noindent Comparing the lift and drag with the diffuser varying with or without an inlet angle, the graphs exhibit an identical results except at 0 degrees. Here, there is a noticeable drop in downforce and drag, which is plausibly due to the diffuser's absence. A well-designed diffuser is important because it slowly expands the flow and allows the jet to stay attached to the undertray surface. Moreover, the side skirts in the diffuser allows the generation of corner vortices that help the flow to remain attached still further, hence increasing the overall downforce. ANSYS Post CFD allows the visualisation of $Q$-Criterion isosurfaces, which marks a vortex region ($Q > 0$) as where the anti-symmetric component of velocity gradient tensor is more dominating than the symmetric \cite{Holmen2012MethodsIdentification}. The corner vortex allows the boundary layer to stick on to the wall, slowing the pressure expansion hence the reduction in lift and drag. This phenomenon can be seen on Figure~\ref{fig:3D_OF_COMPARE_FENCES_SHEAR}, where the region in which the vortex is present has a higher wall shear stress than its surroundings, indicating flow attachment in the respective direction.

\begin{figure}[!htb]
    \centering
    \noindent\makebox[\textwidth]{
    \includegraphics[width=0.65\textwidth]{Figures/3D_OF/3D_OF_COMPARE_FENCES.png}}
    \caption{Comparison of x-wall shear and vortex velocity ($Q$-criterion isosurface) generated on the diffuser region with (left) and without (right) fences applied.}
      \label{fig:3D_OF_COMPARE_FENCES_SHEAR}
\end{figure}

\noindent The next analysis incorporates fences (vertical partitions on the diffuser) to generate corner vortices to investigate their effects on the undertray's performance. The lift and drag plots shown in Figure~\ref{fig:3D_OF_PLOT_COMPARE_ALL} demonstrate an overall higher downforce compared to the bluff body without the fences. A similar concept of vortex generation applies. With the fences installed, there are more vortices generated along the diffuser. This reduces the flow separation in the x direction (flow direction), and maintains flow attachment, thus increasing the downforce. Figure~\ref{fig:3D_OF_COMPARE_FENCES_SHEAR} illustrates the presence of the extra vortices generated inside the diffuser by additional fences, and the smaller flow separation region indicated by non zero x-wall shear region. 

%TALK ABOUT SYMMETRY LIMITATION

\subsection{3D Undertray}
Overall previous analyses have given a fundamental understanding of flow behaviour, features, and performance trends of an underbody flow. This section of the report will utilise prior simulations to design several undertray prototypes that will be simulated using traced bluff-body of QFR car to achieve a sensible picture of the flow in real-life circumstances. The results documented then analysed thoroughly.

\subsubsection{Geometry and Mesh Generation}
\noindent Previous results have developed the knowledge of the trends in flow behaviour for the undertray's inlet and diffuser angles. Eight prototype designs were made based on interpreting these results and some engineering judgement from prior results. A number of configurations were used to achieve the best results in the designs, including: side diffuser; side flat plate; dual variable diffuser; Gurney flaps; and variable vortex generators. These variables were all aimed at achieving the maximum performance of the undertray. All eight undertray prototype geometries can be seen in Figure~\ref{fig:UTP_D} in Appendix D.

%Explain the undertray geometry

\noindent An earlier project by McClune \cite{McClune2018DesignCar} analysed several undertray designs. However, the simulations undertaken consisted solely of the undertray, hence, unrealistic velocity and pressure fields were developed up on the top of the undertray. To overcome this issue in this project, bluff body traced from the real car was developed to approximate the flow over the real QFR 2021 car. Nonetheless, a number of simplifications were made for ease and computational performance, such as the suspension and tyres, which may disturb or pre-sculpt the oncoming flow prior to its entering the side undertray. Figure~\ref{fig:3D_UT_BB_SIMPLIFICATION} shows the QFR car simplification into a bluff body which fits above the undertray design.

\begin{figure}[!htb] 
    \centering
    \noindent\makebox[\textwidth]{
    \includegraphics[width=0.8\textwidth]{Figures/UTP_FIGS/UT_BB_SIMPLIFY.png}}
    \caption{Simplification of QFR car as an undertray's Bluff Body to simulate realistic flow around the body.}
      \label{fig:3D_UT_BB_SIMPLIFICATION}
\end{figure}

 \noindent Fluid-flow then generated using Design Modeller. An partial model enclosed fluid-flow region was made with dimensions of 8 meters behind, 4 meters forward, and 1.5 meters height and thickness. To capture flow feature in the undertray and bluff-body region, two Body of Influence (BOI) were incorporated. A bigger BOI with dimensions of 5 $\times$ 1.2 $\times$ 1.2 meters was generated to capture the flow surrounding the bluff body, and additional smaller BOI on the undertray region used 3 $\times$ 0.25 $\times$ 0.7 meters of dimension. The enclosed fluid flow generated can be seen on \textbf{figure \ref{fig:UTP_Fluid_flow} in Appendix D}
 
 \begin{figure}[!htb] 
    \centering
    \noindent\makebox[\textwidth]{
    \includegraphics[width=0.9\textwidth]{Figures/UTP_FIGS/3D_UT_BB_MESH_COMPILE.png}}
    \caption{3D hybrid mesh generated for undertray prototype design with bluff body.}\label{fig:3D_UT_MESH}
\end{figure}

\noindent A hybrid mesh for 3D Undertray Prototype design is shown in figure \ref{fig:3D_UT_MESH} above. To capture details on the undertray region, 2 BOI were used on both car's surroundings and undertray area. The first BOI used 0.05 meters of sizing and 0.008 for the second BOI near the undertray region with cell growth rate of 1.1. Similarly to other analyses, inflation layers were used on the moving-floor and bluff body. $y^+$ = 1 was used near the body, with configurations of 8 inflation layer, first cell height of 0.00115 meters, and 1.1 cell growth rate. However, to reduce the jump cell size on the far field and still capture the boundary layer in the undertray region, smooth transition inflation layer was used on the floor region. smooth transition of 0.272 (default) with 5 inflation later and 1.1 cell growth rate were used. In results, around 9.5 to 9.8 millions elements were generated with average element quality of 0.785, skewness of 0.206, and aspect ratio of 2.399 which put the overall quality of the mesh in very good region \cite{Ansys2006ModelingFlows}.   

\subsubsection{Results and Discussion}
\textbf{Table \ref{UTB_RESULTS}} below shows the table of eight undertray simulation results to compare with similar simulation environment and settings. The bluff body alone was also simulated to be the base comparison to the one with the undertray attached. It is worth noting that UTP stands for Undertray Prototype as a name simplifications in file and simulations.


\begin{table}[!htb]
\centering
\caption{Simulation Results of Undertray Prototype Designs.}\label{UTB_RESULTS}
\begin{tabularx}{0.95\textwidth}{ 
  | >{\centering\arraybackslash}X 
  | >{\centering\arraybackslash}X
  | >{\centering\arraybackslash}X
  | >{\centering\arraybackslash}X
  | >{\centering\arraybackslash}X
  | >{\centering\arraybackslash}X |
  }
\hline
\multirow{2}{*}{Design Name} & \multicolumn{2}{>{\hsize=\dimexpr2\hsize+2\tabcolsep+\arrayrulewidth\relax\centering}X|}{Full Body Results}  & \multirow{2}{*}{L/D Ratio} & \multicolumn{2}{>{\hsize=\dimexpr2\hsize+2\tabcolsep+\arrayrulewidth\relax\centering}X|}{Aerodynamics Improvement} \\ \cline{2-3} \cline{5-6}
 & Lift (N) & Drag (N) & & Lift (N) & Drag (N) \\
\hline

Bluff Body (Baseline)& -38.48 & 78.72 & -0.49 & 0 & 0\\
\hline
UTP0 & -106.26 & 59.81 & -1.78 & -67.79 & -18.91\\
\hline
UTP1 & -124.00 & 80.78 & -1.53 & -85.50 & 2.06\\
\hline
UTP2 & -222.59 & 68.23 & -3.26 & -184.12 & -10.49\\
\hline
UTP3 & -108.40 & 67.15 & -1.61 & -69.92 & -11.57\\
\hline
UTP4 & -136.75 & 72.60 & -1.88 & -98.28 & -6.12\\
\hline
UTP5 & -225.83 & 80.50 & -2.81 & -187.36 & 1.78\\
\hline
UTP6 & -224.23 & 81.00 & -2.77 & -185.76 & 2.28\\
\hline
UTP7 & -226.52 & 79.77 & -2.84 & -188.05 & 1.05\\
\hline \hline
UTP2 at 20 km/h & -26.06 & 7.77 & -3.35 & N/A & N/A\\
\hline
UTP2 at 100 km/h & -711.31 & 197.74 & -3.60 & N/A & N/A\\
\hline
\end{tabularx}
\end{table}

\noindent The bluff body acts as a baseline which used as a performance improvement indications compared to one with undertray prototype. To get an overall performance variable, efficiency can be deduced from lift to drag ratio in this section. The design process of the undertrays was generative, however some experimental designs were utilised. Several patterns can be seen based on the undertray's design features. It can be seen that all diffuser is shown to have improved the car's performance compared to the plain bluff body. Comparing UTP 0, 3, 4 to other geometries shows that side-diffuser increase the overall downforce but not necessarily the aerodynamics efficiency. For instance, UTP1 has higher downforce compared to UTP0 or UTP4, however UTP 1 has the highest value of aerodynamics efficiency. The results also shown that undertrays with diffuser variables (e.g. UTP4, UTP5, UTP6) have a higher downforce but also directly proportional to its drag. An additional features such as dual-variable diffuser angle and gurney flap such as in UTP3-UTP7 does increase the performance slightly, however the manufacturing complexity made it unnecessary to be included.  Worth mentioning that curved side-diffuser (side diffuser which follow the curve of the body) is the most prominent features that significantly improve the aerodynamics efficiency. It clearly seen that UTP2, UTP5, UTP6, and UTP7 values have almost doubled in efficiency due to the curved side diffuser presence. Based on the L/D ratio, it was decided that undertray prototype 2 (UTP2) will be used as the QFR car's undertray. UTP2 has 0.61 meters of 12 degrees of diffuser (including 6 linearly spaced vortex diffuser) and 1.1 meters side-diffuser with 10 degrees inlet and outlet angle. Other designs such as UTP5-7  have a slightly higher downforce however it also generated high value of drag which made their lift to drag ratio lower compared to UTP2. 

\begin{figure}[!htb] 
    \centering
    \noindent\makebox[\textwidth]{
    \includegraphics[width=1\textwidth]{Figures/UTP_FIGS/UTP_QCRIT_WSHEAR_COMPILE.png}}
    \caption{Undertray Prototype 2 with Bluff Body X Wall Shear and Velocity Contour of Q-Criterion of 0.005 .}
      \label{fig:3D_QCRIT_WSHEAR_UTP2}
\end{figure}

\textbf{Figure \ref{fig:3D_QCRIT_WSHEAR_UTP2}} shows the X wall shear on the bluff body and vortices indicated by the Q-Criterion. Similar to all previous analysis, to improve the lift 

















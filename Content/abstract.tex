\newpage
\thispagestyle{empty}
\section*{Abstract} 
\addcontentsline{toc}{section}{Abstract}
\justifying

\textit{Research into undertray devices in the high-speed racing industry is still privileged information --- this is indicated by the lack of research papers which are forthcoming in the field. In this project, the downforce and drag of an undertray were investigated in two and three dimensions by employing Computational Fluid Dynamics (CFD). Three sets of analyses were undertaken, consisting of 2D enclosed flow, 2D open flow, and 3D open flow, and these were followed by calculations on prototype undertrays to gain a fundamental understanding of the effect of an undertray's flow features on its performance in diverse geometrical setups. Two dimensional simulation was determined not to be optimal for undertray analysis, as it does not properly capture important flow features, which turn out to require three dimensional calculations. The three dimensional analysis has also shown the importance of strategic management of effective flow control using generation of vortices to maintain flow attachment on the undertray's wall and to generate stable vortices to exploit the flow, improving the downforce and reducing drag. The final undertray design geometry was based on these analyses. The final result was found to give a 184.1 N increase in downforce and 10.5 N reduction in drag at 40 $km/h$. Further research into key flow features and vortex generation and their effect on the undertray overall performance is required to utilise the potential performance of an undertray fully.}


\section*{Project Aims \& Objectives}
\addcontentsline{toc}{section}{Project Aims \& Objectives}
This project aims to design and optimise the aerodynamic undertray for the Queen's Formula Racing car, to improve the car's overall downforce and to reduce its drag. The undertray's final design will be based on new computational simulations, as well as design recommendations from previous QFR analyses. The approved final design will then be manufactured and installed onto QFR 2021 for the competition, if workshop time and capacity allows.

\noindent
The key objectives to fulfil this aim are:
\begin{itemize}

    \item Analyse enclosed and open flow 2D \& 3D geometries with various inlet, outlet, and ground clearance variables using ANSYS Fluent, in order to identify their effects on the undertray lift and drag trends. 
    
    \item Design flexible 3D undertrays based on the prior 2D and 3D analysis results, which will again be optimised with additional aerodynamic features on the undertray, improving the downforce and reducing the drag.
	
    \item Manufacture and fit the undertray for the 2021 Queen’s Formula Racing car, which then will be judged in the Formula Student competition.
\end{itemize}
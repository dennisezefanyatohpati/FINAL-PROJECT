\newpage
\thispagestyle{empty}
\section*{Abstract} 
\addcontentsline{toc}{section}{Abstract}
\justifying
Research of an undertray device in the high-speed racing industry is still a privilege information; this indicated by lack of open research paper in the field. Downforce and drag of an undertray in 2 dimensional and 3 dimensional were investigated in this paper by employing Computational Fluid Dynamics (CFD) method. The analyses consist of 2D enclosed, 2D open-flow, 3D open-flow, and undertray prototypes to gain the fundamental understanding the effect of undertray's flow features and performance behaviour in diverse geometrical setup. 2 dimensional simulation decided to be not optimal for undertray analysis as it does not properly capture the flow features in 3 dimensional manner. The 3 dimensional analysis has also shown the importance of strategy and management of effective flow control using optimised vortex generator to maintain flow attachment on the undertray's wall and generate stable vortices to exploit the flow feature, therefore improving the downforce and reducing drag. The final undertray design geometry was based from the previous analysis, the final result indicates 678.5\% improvement in downforce and 13\% reduction in drag at 40 $km/h$. Further research in flow features such as vortex generation, visualisation, and its effect to the undertray overall performance is required to fully expand the potential of an undertray performance. 
